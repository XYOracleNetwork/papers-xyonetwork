% Preamble
% ---
\documentclass{article}

% Packages
% ---
\usepackage{amsmath} % Advanced math typesetting
\usepackage[utf8]{inputenc} % Unicode support (Umlauts etc.)
\usepackage{hyperref} % Add a link to your document
\usepackage{graphicx} % Add pictures to your document
\usepackage{listings} % Source code formatting and highlighting
\usepackage{framed} % Source code formatting and highlighting
\usepackage{appendix} % Source code formatting and highlighting
\usepackage{csquotes} % Pretty quotes
\usepackage[automake]{glossaries}
\usepackage{xcolor}
\usepackage{pagecolor}
\usepackage[letterpaper, portrait, margin=1.5in]{geometry}

\graphicspath{ {images/} }

\makeglossary

%*******************************
%**** Begin Glossary Section *****
%*******************************

\definecolor{lightred}{rgb}{1,0.85,0.85}

\title {XYO Network: Security Risks and Mitigants}

\author{
    Arie Trouw
        \thanks{XYO Network, \texttt{arie@xyo.network}}
}

\date{February 2018 - Early Draft}

\begin{document}

\pagecolor{lightred}

\maketitle

\begin{center}
\line(1,0){50}
\end{center}

\section{Introduction}
A primary concern for the XYO Network, like all decentralized, trustless systems, is security of the system.  Vulnerabilities include, but are not limited to, design/architecture flaws, coding errors, incorrect economic motivation, and social engineering. For the purposes of this document, we will focus on design/architecture flaws and economic motivation.

\section{Pee In The Pool Attacks}
A Pee In The Pool Attack is where a malfunctioning or malicious party is creating corrupt data which decreases the accuracy and/or certainty of results generated by the system.

\section{Assassination Attacks}
An Assassination Attack is where a malicious actor tries to discredit (character assassination) or make non-functional (technical assassination) another node

\section{Deception Attacks}
An Deception Attack is where a malicious actor tries to pass off incorrect yet valid data to be used in the system for personal gain.

\section{Denial of Service Attacks}
An Denial of Service Attack is when a malicious or dysfunctional actor causes a local, regional, or system wide outage.

\section{Revision Attacks}
A Revision Attack is when a malicious or dysfunctional actor causes historically immutable data to mutate.

\end{document}
