% Preamble
% ---
\documentclass{article}

% Packages
% ---
\usepackage{amsmath} % Advanced math typesetting
\usepackage[utf8]{inputenc} % Unicode support (Umlauts etc.)
\usepackage{hyperref} % Add a link to your document
\hypersetup{
    colorlinks=true,
    linkcolor=black,
    filecolor=black,
    citecolor=blue,
    urlcolor=blue,
}
\usepackage{graphicx} % Add pictures to your document
\usepackage{listings} % Source code formatting and highlighting
\usepackage{framed} % Source code formatting and highlighting
\usepackage{appendix} % Source code formatting and highlighting
\usepackage{csquotes} % Pretty quotes
\usepackage{xcolor} % Color Constants
\usepackage{pagecolor} % Page Background
\usepackage[automake]{glossaries}
\usepackage[letterpaper, portrait, margin=1.5in]{geometry}

\graphicspath{ {images/} }

\makeglossaries

%*******************************
%**** Begin Glossary Section *****
%*******************************

\newglossaryentry{sentinel}
{
    name={Sentinel},
    description={A Sentinel is a heuristic witness. It observes heuristics and vouches for the certainty and accuracy of them by producing temporal ledgers. The most important aspect of a Sentinel is that it produces ledgers that Diviners can be certain came from the same source by adding Proof of Origin to them}
}

\newglossaryentry{bridge}
{
    name={Bridge},
    description={A Bridge is a heuristic transcriber. It securely relays heuristic ledgers from Sentinels to Diviners. The most important aspect of a Bridge is that a Diviner can be sure that the heuristic ledgers that are received from a Bridge have not been altered in any way. The second most important aspect of a Bridge is that they add an additional Proof of Origin metadata}
}

\newglossaryentry{archivist}
{
    name={Archivist},
    description={An Archivist stores heuristics as a part of the decentralized data set with the goal of having all historical ledgers stored, but without that requirement. Even if some data is lost or becomes temporarily unavailable, the system continues to function, just with reduced accuracy. Archivists also index ledgers so that they can return a string of ledger data if needed. Archivists store raw data only and get paid solely for retrieval of the data. Storage is always free}
}

\newglossaryentry{diviner}
{
    name={Diviner},
    description={A Diviner answers a given query by analyzing historical data that has been stored by the XYO Network. Heuristics stored in the XYO Network must have a high level of Proof of Origin to determine the validity and accuracy of the heuristic. A Diviner obtains and delivers an answer by judging the witness based on its Proof of Origin. Given that the XYO Network is a trustless system, Diviners must be incentivized to provide honest analyses of heuristics. Unlike Sentinels and Bridges, Diviners use Proof of Work to add answers to the blockchain}
}

\newglossaryentry{node}
{
    name={Node},
    description={TODO}
}

\newglossaryentry{module}
{
    name={Module},
    description={TODO}
}

\newglossaryentry{bound-witness}
{
    name={Bound Witness},
    description={Bound Witness is a concept achieved by the existence of a bidirectional heuristic. Given that an untrusted source of data for the use of digital contract resolution (an oracle) is not useful, there is a substantial increase in certainty of the data provided by the establishment of such a heuristic. The primary bidirectional heuristic is proximity since both parties can validate the occurrence and range of an interaction by cosigning the interaction. This allows for a zero-knowledge proof that the two nodes were in proximity of each other.}
}

\newglossaryentry{smart-contract}
{
    name={smart contract},
    description={A protocol coined by Nick Szabo before Bitcoin, purportedly in 1994 (which is why some believe him to be Satoshi Nakamoto, the mystical and unknown inventor of Bitcoin). The idea behind smart contracts is to codify a legal agreement in a program and to have decentralized computers execute its terms, instead of humans having to interpret and act on contracts. Smart contracts collapse money (e.g. Ether) and contracts into the same concept. Being that smart contracts are deterministic (like computer programs) and fully transparent and readable, they serve as a powerful way to replace middle-men and brokers}
}

\newglossaryentry{cryptoeconomics}
{
    name={cryptoeconomics},
    plural={cryptoeconomic},
    description={A formal discipline that studies protocols that govern the production, distribution, and consumption of goods and services in a decentralized digital economy. Cryptoeconomics is a practical science that focuses on the design and characterization of these protocols}
}

\newglossaryentry{xyo-network}
{
    name={XYO Network},
    description={XYO Network stands for ``XY Oracle Network.'' It is comprised of the entire system of XYO enabled components/nodes that include Sentinels, Bridges, Archivists, and Diviners. The primary function of the XYO Network is to act as a portal by which digital smart contracts can be executed through real world geo-location confirmations}
}

\newglossaryentry{certainty}
{
    name={certainty},
    description={A measure of the likelihood that a data point or heuristic is free from corruption or tampering}
}

\newglossaryentry{accuracy}
{
    name={accuracy},
    description={A measure of confidence that a data point or heuristic is within a specific margin of error}
}

\newglossaryentry{oracle}
{
    name={oracle},
    description={A part of a DApp (decentralized application) system that is responsible for resolving a digital contract by providing an answer with accuracy and certainty. The term ``oracle'' originates from cryptography where it signifies a truly random source (e.g. of a random number). This provides the necessary gate from a crypto equation to the world beyond. Oracles feed smart contracts information from beyond the chain (the real world, or off-chain). Oracles are interfaces from the digital world to the real world. As a morbid example, consider a contract for a Last Will \& Testament. A Will's terms are executed upon confirmation that the testator is deceased. An oracle service could be built to trigger a Will by compiling and aggregating relevant data from official sources. The oracle could then be used as a feed or end-point for a smart contract to call out to in order to check whether or not the person is deceased}
}

\newglossaryentry{heuristic}
{
    name={heuristic},
    description={A data point about the real world relative to the position of a Sentinel (proximity, temperature, light, motion, etc...)}
}

\newglossaryentry{trustless}
{
    name={trustless},
    description={A characteristic where all parties in a system can reach a consensus on what the canonical truth is. Power and trust is distributed (or shared) among the network’s stakeholders (e.g. developers, miners, and consumers), rather than concentrated in a single individual or entity (e.g. banks, governments, and financial institutions). This is a common term that can be easily misunderstood. Blockchains don’t actually eliminate trust. What they do is minimize the amount of trust required from any single actor in the system. They do this by distributing trust among different actors in the system via an economic game that incentivizes actors to cooperate with the rules defined by the protocol}
}

\newacronym{poo}{PoO}{Proof of Origin}

\newacronym{xy-oracle-network}{XY Oracle Network}{XYO Network}

\title {XY0 2.0 Platform: The Sovereign Internet Platform based on the XYO Protocol}

\author{
    Arie Trouw
        \thanks{XYO Network, \texttt{arie.trouw@xyo.network}}
    , Joel Carter
        \thanks{XYO Network, \texttt{joel.carter@xyo.network}}
    , Matt Jones
        \thanks{XYO Network, \texttt{matt.jones@xyo.network}}
}

\date{January 2024}

\begin{document}
\pagecolor{yellow!25}
\maketitle

\begin{center}
    \line(1,0){50}
\end{center}

%Abstract Section
\begin{abstract}
    The XYO 2.0 Platform is a system implementation of the XYO Protocol as defined in the XYO Protocol Whitepaper published in January 2018.  It focuses on providing a solution that achieves high performance without sacrificing the sovereignty, provenance, and permanence that is the goal setout by the whitepaper.  This XYO 2.0 Platform also expands the usage of the core concepts defined in the White Paper to be useful in a much broader set of use-cases, specifically not limiting its use to location.  The implementation set forth in this Yellow Paper adds additional protocol definitions to provide guidelines through which future components and alternative implementations can be created while maintaining the ability for them to work together to form a singular XYO Network.
    \begin{center}
        \line(1,0){50}
    \end{center}
\end{abstract}

%Introduction
\section{Introduction}
During the last decade, Web 3 development has been primarily focused on expanding the use of shared ledgers to create decentralized systems.  Even though there have been great strides on this front, the very core of this effort is flawed in two ways.

First, shared ledgers moves the control of a system from being fully centralized (effectively a kingdom model) towards a decentralized solution that is based on majority rule and finality (effectively a democracy or republic).  Like with all governance systems, the natural evolution of these systems have pulled back from maximizing decentralization towards more centralized concepts for practical, regulatory, or other, potentially sinister, reasons.  Even if this pull-back did not occur, the ceiling of shared ledger decentralization is that of majority rule and not true sovereignty.

Second, the performance of shared ledger technology has been and will always be substantially (orders of magnitude) slower and more costly than their centralized equivalents.  By its very definition, a shared ledger must either has massive redundancy of data and validation, or lean on trusted systems to improve performance.

This implementation of the XYO 2.0 Platform combined with the core concepts of the XYO Protocol strives to provide full decentralization with nodes that are 100 percent sovereign while using cryptographic technologies and concepts to deliver a trustless network that has performance at scale comparable or better than the performance of an equivalent Web 2 system and orders of magnitude better than equivalent Web 3 systems.  This combination not only delivers on the goals of Web 3 visionaries, but also delivers on the goals set forth by the original founders of the internet. The current Web 2 implementation of the internet is completely devoid of sovereignty, provenance, and permanence and we must reverse that trend by delivering a solution that is the foundation for the Sovereign Internet by combining the core tenants of Web 2 and Web 3 combined with the concepts of the XYO Protocol as set out in the original XYO White Paper.  

\clearpage
\section{Practical Decisions}
In producing the XYO 2.0 Platform, various practical decisions have been made to facilitate interoperability and reduce ambiguity in the protocol.

\subsection{Programming Language}
The initial version of the XYO 2.0 Platform has been developed using TypeScript.  We chose this due to the expansive tools that are available for developing with TypeScript and the compatibility that Javascript (the output of compiling TypeScript) allows for.  As a result, this implementation can be used on browsers and with the NodeJS runtime, both on desktop and on mobile devices.  The sacrifice of this decision is that running the technology stack on IoT devices, especially battery powered devices will be negatively impacted.  This can be addressed by creating limited native implementations of the Platform for those devices.

WebAssembly is used for various high performance cryptographic algorithms since WebAssembly can be seamlessly interacted with from Javascript.  Over time, it is possible that more TypeScript based code could be replaced with WebAssembly, but that will be done with care since there are costs of doing this when it comes to understandability of code and debugging.

\subsection{Payload Encoding}

\subsection{BoundWitness Encoding}

\subsection{Hashing}

\subsection{Addresses}

\subsection{Signing}

\subsection{Big Numbers}

\section{Module System}

\subsection{Manifests}

\subsection{Addressing}

\subsection{Discovery}

\subsection{Eventing}

\section{Component System}
We chose React as the framework to create user-interface components for the XYO 2.0 Platform.  This does not preclude using other frameworks and in the cases where native user-interfaces are required, alternative frameworks will be required.  In all these cases, the paradigm which we embrace in our React implementation should be followed.

\subsection{Renderers}
A Renderer is a component that takes a working set of payloads and renders them.  In many cases, the renderer is so simple that it only has the ability to render a single payload of a specific type.  The most basic of renderers simply renders a payload's raw data, which usually is a JSON object.  In most cases, a dApp strives to have much friendlier renderers than the basic renderer.  Currently, the only way to make custom renderers is to create a new React component usually based on an existing renderer such as the basic raw renderer.  We have a future goal to produce a codeless system to producing renderers.   

\subsection{Hooks}

Hooks are a specific system used in React, but as a paradigm, exists in most user-interface frameworks.  The purpose of hooks to to separate the acquisition/manipulation of data from the rendering of that data.  In most cases, the provided generic hooks that provide access to XYO modules are sufficient to gather the data in a user interface for a dApp.

\section {Codeless Development}
A primary goal of the XYO 2.0 Platform is to allow users to fully customize nodes through Codeless Development using the module system with manifests, configurations, and parameters. The only reason to create custom modules using TypeScript directly should be for performance reasons, not because what is being created is not possible via the Codeless Development paradigm supported by the platform.

\section {Security}
Security concerns in the XYO Platform follows that of the XYO Protocol exclusively using cryptographic mechanisms to provide security.  This includes hashes, addresses, and cryptographic signing of data.  As for individual security on specific modules, it is left up to that module to mange access either through lists of allowed/disallowed addresses, crypto-economic incentives, or a combination of the two.

\section {Privacy}
Privacy itself is not addressed in the core XYO Protocol, however, the ability to optionally keep payloads private is accommodated by the protocol.  There are two primary paradigms that we utilize in the XYO 2.0 Platform.

\subsection{Just-In-Time Privacy}
Just-In-time Privacy (JITP) for XYO is the paradigm where a node shares hashes of payloads without sharing the actual payloads. This allows sovereign Bound Witnesses to be created, establishing provenance and maintaining immutable permanence, while maintaining privacy of the originating payload.  JITP can be used to create sovereign games, where multiple parties can lock in moves without exposing what those moves are, and then only exposing those moves when declaring victory.

\subsection{Subnet Privacy}
Subnet Privacy is the ability to run a private XYO Subnet.  This is accomplished by running one or more nodes that are networked together, allowing each other to have access that is not available to nodes outside of that network.  This is similar to a traditional intranet.  The security concerns are also very similar to intranets in that there will be nodes that have access to both the private XYO Subnet and to external networks such as other private subnets or the public XYO Network.  Since those nodes are on multiple networks at the same time, it is possible for them to intentionally or accidentally leak private data from the private subnet to the external network, so great care must be taken when allowing nodes to connect to a private subnet.

% Acknowledgements
\section {Acknowledgements}
This white paper is the product of an inspiring team effort that was made possible through the belief in our vision from the following individuals: 

\begin{center}
    \line(1,0){50}
\end{center}


% TODO: Christine - DONE

\begin{thebibliography}{9}

    \bibitem{xyo-protocol}
    Trouw, Arie; Levin, Markus; Sheper, Scott
    \textit{XYO Protocol White Paper}.
    XYO Website. January 2018

\end{thebibliography}

\clearpage

\printglossaries

%*******************************
%**** End Glossary Section *****
%*******************************

\end{document}
